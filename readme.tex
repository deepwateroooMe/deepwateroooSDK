% Created 2022-12-03 Sat 09:29
\documentclass[9pt, b5paper]{article}
\usepackage{xeCJK}
\usepackage{minted}
\usepackage[T1]{fontenc}
\usepackage[scaled]{beraserif}
\usepackage[scaled]{berasans}
\usepackage[scaled]{beramono}
\usepackage{graphicx}
\usepackage{xcolor}
\usepackage{multirow}
\usepackage{multicol}
\usepackage{float}
\usepackage{textcomp}
\usepackage{algorithm}
\usepackage{algorithmic}
\usepackage{latexsym}
\usepackage{natbib}
\usepackage{geometry}
\geometry{left=1.2cm,right=1.2cm,top=1.5cm,bottom=1.2cm}
\newminted{common-lisp}{fontsize=\footnotesize} 
\usepackage[xetex,colorlinks=true,CJKbookmarks=true,linkcolor=blue,urlcolor=blue,menucolor=blue]{hyperref}
\author{deepwaterooo}
\date{\today}
\title{游戏通用功能底层逻辑Android SDK 封装}
\hypersetup{
  pdfkeywords={},
  pdfsubject={},
  pdfcreator={Emacs 27.2 (Org mode 8.2.7c)}}
\begin{document}

\maketitle
\tableofcontents


\section{Android SDK 封装for unity games}
\label{sec-1}
\begin{itemize}
\item Initial design, to be modified and linked later
\item Will work on packing lower layer Android SDK feature/functionality packing for coming a few days
\item 屏幕适配可需要再改一下
\end{itemize}

\section{Unity安卓共享纹理}
\label{sec-2}
\begin{itemize}
\item 之前一直有想法:就是把游戏中所用到的平移(四个按钮)与旋转(六个按钮)做成安卓原生开发,做成透明,与游戏端画面叠加绘制
\begin{itemize}
\item 原本也可以不用,前提是我能够在游戏端将这两块画面做得漂亮
\item 自己做不漂亮的时候,就去想,能不能从安卓端原生画出来,毕竟画出来的刻度会是精准的,并将十个按钮设置为半透明,达到精准完美的程度
\end{itemize}
\item 现在看,这些实现起来都是有理论支撑,可以做到的,参考的github项目也已经被我fork到自己的了
\item 等安卓SDK接完,可能会试一下自己游戏中能够实现这些
\item 一些基础原理:
\end{itemize}
\begin{minted}[fontsize=\scriptsize,linenos=false]{java}
public class GLTexture {
    private static final String TAG = "GLTexture";

    private static final String imageFilePath = "/sdcard/1Atest/image.jpg";

    private int mTextureID = 0;
    private int mTextureWidth = 0;
    private int mTextureHeight = 0;

    SurfaceTexture mCameraInputSurface;
    SurfaceTexture mOutputSurfaceTexture;
    int mOutputTex[];

// OpenGL 渲染的上下文及配置: 多线程安全(安卓 游戏)
    private volatile EGLContext mSharedEglContext;
    private volatile EGLConfig mSharedEglConfig;

    private EGLDisplay mEGLDisplay;
    private EGLContext mEglContext;
    private EGLSurface mEglSurface;

    // 创建单线程池,用于处理OpenGL纹理
    private final ExecutorService mRenderThread = Executors.newSingleThreadExecutor();
    // 使用Unity线程Looper的Handler,用于执行Java层的OpenGL操作
    private Handler mUnityRenderHandler;

    public GLTexture() { }
    public int getStreamTextureWidth() {
        //Log.d(TAG,"mTextureWidth = "+ mTextureWidth);
        return mTextureWidth;
    }
    public int getStreamTextureHeight() {
        //Log.d(TAG,"mTextureHeight = "+ mTextureHeight);
        return mTextureHeight;
    }
    public int getStreamTextureID() {
        Log.d(TAG,"getStreamTextureID sucess = "+ mTextureID);
        return mTextureID;
    }
    private void glLogE(String msg) {
        Log.e(TAG, msg + ", err=" + GLES20.glGetError());
    }

    // 被unity调用
    public void setupOpenGL() {
        Log.d(TAG, "setupOpenGL called by Unity ");

        // 注意:该调用一定是从Unity绘制线程发起
        if (Looper.myLooper() == null) {
            Looper.prepare();
        }
        mUnityRenderHandler = new Handler(Looper.myLooper());

        // Unity获取EGLContext
        mSharedEglContext = EGL14.eglGetCurrentContext();
        if (mSharedEglContext == EGL14.EGL_NO_CONTEXT) {
            glLogE("eglGetCurrentContext failed");
            return;
        }
        glLogE("eglGetCurrentContext success");

        EGLDisplay sharedEglDisplay = EGL14.eglGetCurrentDisplay();
        if (sharedEglDisplay == EGL14.EGL_NO_DISPLAY) {
            glLogE("sharedEglDisplay failed");
            return;
        }
        glLogE("sharedEglDisplay success");

        // 获取Unity绘制线程的EGLConfig
        int[] numEglConfigs = new int[1];
        EGLConfig[] eglConfigs = new EGLConfig[1];
        if (!EGL14.eglGetConfigs(sharedEglDisplay, eglConfigs, 0, eglConfigs.length,
                                 numEglConfigs, 0)) {
            glLogE("eglGetConfigs failed");
            return;
        }
        mSharedEglConfig = eglConfigs[0];
        mRenderThread.execute(new Runnable() {
                @Override
                public void run() {
                    // 初始化OpenGL环境
                    initOpenGL();
                    // 生成OpenGL纹理ID
                    int textures[] = new int[1];
                    GLES20.glGenTextures(1, textures, 0);
                    if (textures[0] == 0) { glLogE("glGenTextures failed"); return; }
                    else { glLogE("glGenTextures success"); }
                    mTextureID = textures[0];
                    mTextureWidth = 670;
                    mTextureHeight = 670;
                }
            });
    }
    private void initOpenGL() {
        mEGLDisplay = EGL14.eglGetDisplay(EGL14.EGL_DEFAULT_DISPLAY);
        if (mEGLDisplay == EGL14.EGL_NO_DISPLAY) {
            glLogE("eglGetDisplay failed");
            return;
        }
        glLogE("eglGetDisplay success");

        int[] version = new int[2];
        if (!EGL14.eglInitialize(mEGLDisplay, version, 0, version, 1)) {
            mEGLDisplay = null;
            glLogE("eglInitialize failed");
            return;
        }
        glLogE("eglInitialize success");

        int[] eglContextAttribList = new int[]{
            EGL14.EGL_CONTEXT_CLIENT_VERSION, 3, // 该值需与Unity绘制线程使用的一致
            EGL14.EGL_NONE
        };
        // 创建Java线程的EGLContext时,将Unity线程的EGLContext和EGLConfig作为参数传递给eglCreateContext,
        // 从而实现两个线程共享EGLContext
        mEglContext = EGL14.eglCreateContext(mEGLDisplay, mSharedEglConfig, mSharedEglContext,
                                             eglContextAttribList, 0);
        if (mEglContext == EGL14.EGL_NO_CONTEXT) {
            glLogE("eglCreateContext failed");
            return;
        }
        glLogE("eglCreateContext success");

        int[] surfaceAttribList = {
            EGL14.EGL_WIDTH, 64,
            EGL14.EGL_HEIGHT, 64,
            EGL14.EGL_NONE
        };
        // Java线程不进行实际绘制,因此创建PbufferSurface而非WindowSurface
        // 创建Java线程的EGLSurface时,将Unity线程的EGLConfig作为参数传递给eglCreatePbufferSurface
        mEglSurface = EGL14.eglCreatePbufferSurface(mEGLDisplay, mSharedEglConfig, surfaceAttribList, 0);
        if (mEglSurface == EGL14.EGL_NO_SURFACE) {
            glLogE("eglCreatePbufferSurface failed");
            return;
        }
        glLogE("eglCreatePbufferSurface success");

        if (!EGL14.eglMakeCurrent(mEGLDisplay, mEglSurface, mEglSurface, mEglContext)) {
            glLogE("eglMakeCurrent failed");
            return;
        }
        glLogE("eglMakeCurrent success");

        GLES20.glFlush();
    }
    public void updateTexture() {
        // Log.d(TAG,"updateTexture called by unity");
        mRenderThread.execute(new Runnable() {
                @Override
                public void run() {
                    final Bitmap bitmap = BitmapFactory.decodeFile(imageFilePath);
//                if(bitmap == null)
//                    Log.d(TAG,"bitmap decode faild" + bitmap);
//                else
//                    Log.d(TAG,"bitmap decode success" + bitmap);
                    mUnityRenderHandler.post(new Runnable() {
                            @Override
                            public void run() {
                                GLES20.glBindTexture(GLES20.GL_TEXTURE_2D, mTextureID);
                                GLES20.glTexParameteri(GLES11Ext.GL_TEXTURE_EXTERNAL_OES, GLES20.GL_TEXTURE_MIN_FILTER, GLES20.GL_NEAREST);
                                GLES20.glTexParameteri(GLES11Ext.GL_TEXTURE_EXTERNAL_OES, GLES20.GL_TEXTURE_MAG_FILTER, GLES20.GL_NEAREST);
                                GLES20.glTexParameteri(GLES20.GL_TEXTURE_2D, GLES20.GL_TEXTURE_WRAP_S, GLES20.GL_CLAMP_TO_EDGE);
                                GLES20.glTexParameteri(GLES20.GL_TEXTURE_2D, GLES20.GL_TEXTURE_WRAP_T, GLES20.GL_CLAMP_TO_EDGE);
                                GLES20.glTexParameteri(GLES20.GL_TEXTURE_2D, GLES20.GL_TEXTURE_MAG_FILTER, GLES20.GL_LINEAR);
                                GLES20.glTexParameteri(GLES20.GL_TEXTURE_2D, GLES20.GL_TEXTURE_MIN_FILTER, GLES20.GL_LINEAR);
                                GLUtils.texImage2D(GLES20.GL_TEXTURE_2D, 0, bitmap, 0);
                                GLES20.glBindTexture(GLES20.GL_TEXTURE_2D, 0);
                                bitmap.recycle();
                            }
                        });
                }
            });
    }
    public void destroy() {
        mRenderThread.shutdownNow();
    }
}
\end{minted}
% Emacs 27.2 (Org mode 8.2.7c)
\end{document}